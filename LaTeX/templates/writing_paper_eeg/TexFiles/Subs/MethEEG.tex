%! TEX root = ../main.rnw
EEG activity was continuously recorded from 70
Ag/AgCl electrodes over midline electrodes Fpz, AFz, Fz, FCz, Cz, CPz, Pz, POz,
Oz, and Iz, over the left hemisphere from electrodes IO1, Fp1, AF3, AF7, F1,
F3, F5, F7, F9, FC1, FC3, FC5, FT7, C1, C3, C5, M1, T7, CP1, CP3, CP5, TP7, P1,
P3, P5, P7, O9, PO3, PO7, O1, and from the homologue electrodes over the right
hemisphere using a BIOSEMI Active-Two amplifier system. Two non-standard
electrodes (PO9 and PO10) were positioned at 33\% and 66\% of the M1-Iz
distance (M2-Iz for the right hemisphere). Two additional electrodes (Common
Mode Sense [CMS] active electrode and Driven Right Leg [DRL] passive electrode)
were used as reference and ground electrodes, respectively
(cf., www.biosemi/faq/cms\&drl.htm). EEG and EOG recordings were sampled at 1024 Hz.
Vertical electroocular (vEOG) and horizontal EOG (hEOG) waveforms were
calculated offline as follows: vEOG(t) = Fp1(t) minus IO1(t) and hEOG(t) =
F9(t) minus F10(t). All EEG/ERP analysis was performed using available MATLAB
toolboxes (EEGLAB: \textcite{delorme_eeglab_2004}; FieldTrip:
\textcite{oostenveld_fieldtrip_2011} and custom MATLAB scripts). Continuous EEG and EOG
activity was high-pass filtered (0.1 Hz) and re-referenced to the average of
all electrodes. Data were time-locked to feedback onset (Feedback-Locked). EEG
data processing steps were as follows: First, epochs containing extreme values
from non-neural and non-ocular sources (e.g., amplifier blockings) were
removed, as were trials containing values exceeding ±75 \textmu V in multiple
electrodes that were not related to pro-typical eye movements (e.g., large head
movements). Subsequently, noisy electrodes were identified and excluded by
calculating z-scored variance measures across all electrodes and excluding
electrodes with a value greater than 3 SDs whose activity was uncorrelated to
either hEOG or vEOG activity. ICA was performed on this \enquote{cleaned} EEG
data set via runica \parencite[see][]{delorme_eeglab_2004, makeig_blind_1997} that
implements infomax ICA algorithm \parencite[][]{bell_information-maximization_1995}.
ICA components representing ocular activity (blinks and horizontal eye
movements) were automatically identified using z-scored measures (> 3 SDs) of
the absolute correlation between the ICA component and the recorded hEOG and
vEOG activity (confirmed via visual inspection of ICA component topographies
and time-course). Identified components were subtracted from the EEG dataset.
Previously removed noisy channels (if any) were interpolated using the average
of neighbouring channels within a specified distance (4 cm and 3-4 neighbors per
electrode) in order to ensure a full electrode array for each participant.
Subsequently, artifacts (EEG activity exceeding ±75 \textmu V) were corrected in a
single trial/single electrode correction procedure with the constraint that all
neighbouring electrodes were artifact free. A final stage of artifact rejection
excluded remaining trials with activity exceeding ±75 \textmu V in any electrode. A
similar methodological procedure for the automatic detection and removal of
ocular-based artifacts is implemented in \textcite{nolan_faster_2010}. The
resulting ERP waveforms were low-pass filtered (30 Hz, two-pass 36 dB/octave).
