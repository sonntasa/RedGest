%! TEX root = ../main.rnw
Common conversations are not only defined by speech, but a plethora of other
verbal and non-verbal means of communication such as gestures and facial
expressions.\\
The Stroop- \parencite[][]{stroop_studies_1935}, Simon- \parencite[][]{simon_auditory_1967}
and Flanker task \parencite[][]{eriksen_effects_1974} are arguably some of the
most well-established experimental paradigms in the field of (cognitive)
psychology. Decomposing each task, the unifying element is that there always
exist two dimension to each task that both convey some sort of information.
Within each task, participants are instructed to disregard one target
dimension in favor of the other. Within the Stroop Task that would mean that
participants are instructed to respond to the color of a stimulus. However,
the stimulus is also a color word and therefore conveys a second layer of
information. If both the color and identity of the stimulus elicit the same
response (\textit{are congruent}), participants have repeatedly been shown to
react faster and more accurate than in trials where both dimensions contradict
each other (\textit{incongruent}). This difference in performance (in both
accuracy and reaction times) has since been dubbed the \textit{congruency
	effect} \parencite[][]{botvinick_conflict_2001, botvinick_conflict_2004,
	stroop_studies_1935, simon_auditory_1967, eriksen_effects_1974}.\\
A successful strand of various modern conceptions attempting to account for this
\textit{congruency effect} are the various \textit{dual-route models}
\parencite[e.g.,][]{de_jong_conditional_1994, ridderinkhof_micro-_2002,
	botvinick_conflict_2001, botvinick_conflict_2004} that all hinge upon the
assumption that both relevant and irrelevant information are being processed
in parallel. The influential \textit{conflict monitoring hypothesis}
\parencite[][]{botvinick_conflict_2001} posits the existence of an
independent process monitoring for the occurrence of conflict, translating
it into compensatory adjustments in control. The congruency effect then
supposedly arises when relevant and irrelevant information are being
superimposed. The congruency effect then supposedly arises when relevant and
irrelevant information are being superimposed.\\
