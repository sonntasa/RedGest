%! TEX root = ../main.rnw
Among the most crucial distinctions between humans and other animals are
the sophisticated means of communication humanity has developed. Language, at
its best, allows humanity an unlimited scope of communication, providing the
groundwork for stable social interactions as well as the efficient
organization and exchange of information that builds the basis for cumulative
scientific advancement.
Nonetheless, our understanding of the architecture of our cognitive language
system, the very thing that allows for the accumulation of knowledge and
understanding is not only limited, but has also received surprisingly little
attention in the respective fields of research.

Some of the earliest and perhaps also most influential contributions to the
understanding of the human language apparatus were made by Carl Wernicke
(\citeyear{wernicke_aphasische_1874}) and Paul Broca (\citeyear{broca_classics_1861}), who's
discoveries both hinge upon the assumption of strong anatomical modularity
of the (human) brain. Their respective infamy is both contingent upon the
identification of areas in the brain thought to be responsible for the
comprehension/production of language through the observation of the
double-dissociation between cognitive functionality
and cognitive trauma. While this strictly localist view of language processing
in the brain has long since  made way for the assumption of the interconnection
of different areas and functions within cognitive science, the make-up of
information processing within the brain remains a fruitful field of research
to this day.
One recent theory that accounts for (among others) the make-up of the human
language system is the so-called \textit{neural-reuse hypothesis}
\parencite[][]{anderson_neural_2010}.
It's principle thought is that more complex neurological mechanisms
(i.e.\ the language system) are comprised of the exaptation of (evolutionary)
earlier, more basic mechanisms, often without abandoning their original
functionalities. Importantly, this account is functional and not anatomical in
nature and does not specify the makeup of any certain process.

An offshoot of this hypothesis that has repeatedly been shown in recent work
is the reuse of inhibitory mechanisms in the processing of (sentential)
negation. Several researchers postulate (and provide evidence for) a
structural relationship
between motor inhibition and  the processing of sentential negation in that
they share parts of their processing apparatus. This relationship manifests
itself in the deleterious influence on participants performance in several
tasks that is observed if both motor inhibition and sentential negation are
being processed at the same time \parencite[see:][]{garcia-marco_negation_2019,
	aravena_grip_2012, de_vega_sentential_2016, montalti_is_2023,
	beltran_sentential_2018, beltran_brain_2019}.
Despite the apparent replicability of this effect,
why and how inhibition seems to be a prerequisite for the processing
of negation is yet to be discovered.\\

% Within the current study, the work of both \textcite{montalti_is_2023,
% 	beltran_sentential_2018} is of special interest.
Methodologically, the current study is built upon the work of both
\textcite{montalti_is_2023, beltran_sentential_2018}.
They both employ a
Stop-Signal paradigm combined with the processing of sentential negation to
provide evidence that cognitive inhibitory mechanisms are involved (being
reused) in both motor inhibition and the processing of negation
(by the processing of sentential negation). To do so,
they both have participants read action sentences (made up of two words
for \citeauthor{montalti_is_2023} or whole sentences
for \citeauthor{beltran_sentential_2018}) that are either affirmative or negated.
Both report a deleterious influence of sentential negation on the efficacy of
motor inhibition that manifests itself in a respectively longer
\textit{Stop Signal Reaction Time} (SSRT). How and why both processes seem to
share resources however is a question neither study attempts to answer.

An account that might provide insights into the nature of shared
resources between sentential negation and (motor) inhibition is
the two-step hypothesis \parencite[][]{kaup_processing_2006, dudschig_how_2018}. Its
fundamental premise is that in-order to process negation (i.e. \textit{the
	door is not open}), two states of affairs have to be understood. The state
that is being negated (\textit{the door is open}) and the actual state of
affairs (\textit{the door is not open}). It is herein assumed that in the first step
of information processing, representations of both states of affairs
are being constructed, whereas in the second stage of processing,
the irrelevant representation is being suppressed. The hypothesis thereby
accounts for the exaptation of inhibitory mechanisms in a functional manner.

Reconsidering the design introduced by \textcite{beltran_sentential_2018} and
adapted by \textcite{montalti_is_2023}, a key change will be made in the
following study. In only using action words as the target of sentential
affirmation/negation, it is not possible to distinguish between the effects
that are inherent to operational negation and the negation of a simulated
action. Within the current study, two word stimuli
\parencite[as in][]{montalti_is_2023} will be used. However, not all stimuli
will merit the simulation of a concept. To ensure this, in addition to 10
action words, 10 pseudo words will be presented as the target of the
operational affirmation/negation.\\
If the employment of inhibitory mechanisms in fact stems from the necessary
suppression of a simulated representation of the sentence, pseudo-words that
do not elicit a simulation should mitigate the influence of the
affirmation/negation. Therefore, all markers of motor inhibition should not
differ significantly between the affirmative/negated condition.\\
If on the other hand, the recruitment of inhibition were to be an inherent property of the
processing of negation, no matter the context, we should observe a similar
influence of operational negation, independent of the action/pseudo word used
as a target.

% NOTE: montalti_is_2023 and beltran_sentential_2018 use action
% sentences/words 
% If we were able to show the effect was contingent upon the usage of
% simulation based concepts, this would serve as strong evidence for the
% two-step simulation hypothesis.
% If the effect was in fact not contingent upon the word, but just the
% processing of the negation operator, we would have strong evidence for the
% idea that inhibitory mechanisms are inherently related to the processing of
% negation.

Demonstrating how the processing of sentential negation and motor inhibition
are functionally akin might not only serve to instantiate the
\textit{neural-reuse hypothesis} but also serve as an explanation how the
functional inheritance in \textit{neural-reuse} is practically felicitous.

% As \textcite{montalti_is_2023} argue, if the processing of sentential
% negation and the motor inhibition of a pre-planned movement both compete for
% shared resources, a deleterious influence of sentential negation on the
% efficacy of motor inhibition should be observable. If we were able to reproduce this
% effect while also showing physiological evidence, this study would serve to instantiate the
% \textit{neural-reuse hypothesis} by supplying specific example of the functional
% exaptation of cognitive mechanisms.

% While this bodily grounding of concrete concepts has become a relatively
% unequivocal finding, a recent strand of research has emerged that claims to
% have uncovered a similar mechanism underlying the processing of relatively
% abstract concepts such as sentential negation
% \parencite[][]{beltran_sentential_2018, montalti_is_2023, dudschig_negation_2021}\\

The Stop-Signal task that both papers employ as an experimental paradigm is
a relatively novel one in the context of psycholinguistics, but a well
established measure of inhibitory control in the realm of experimental
cognitive psychology that offers a range of possible metrics.\\

\subsection{Experiment 1}
\label{sub:experiment_1}
In the following study we aim to enhance our understanding of the processing
of sentential negation by utilizing the physiological measure of response
force exerted by our participants. In the Stop-Signal paradigm, participants
perform a simple Go-Task that requires them to react to an unambiguous target
cue. On some trials however, participants receive an additional (auditory)
stop cue that requires them to inhibit their response. According to the
Horse-race model \parencite[][]{logan_ability_1984, logan_ability_1984-1, band_horse-race_2003}
whether or not an answer is given on trials where a stop signal
occurs is determined by an independent race between the activation elicited by
the Go-target and the inhibitory activation elicited by the Stop-Signal.
Whichever activation finishes the race first is the one that determines the
outcome of each trial.\\

Initial accounts of this hypothesis were based on the assumption of a
dichotomy of outcomes that is determined by the winner of the aforementioned
race that is either a response is given, or no response is given. Recently
however a myriad of studies were able to demonstrate that as like most
dichotomies, this distinction appears to be a vast oversimplification of
the real state of affairs. It has been shown (reliably so) that on trial
where participants had reached the so-called \textit{point of no return} and
had given an erroneous response in the presence of an auditory stop-signal,
the response force was attenuated compared to the response force in correct
responses \parencite[][]{ko_inhibitory_2012, nguyen_go_2021,
	nguyen_motor_2020, weber_stopping_2024, wang_unravelling_2023,
	weissman_proactive_2024, salomoni_proactive_2023}
This demonstrates that there is at least a trichotomy of responses in the
Stop-Signal task. \textit{Correct
	responses}, \textit{successful inhibitions} and \textit{unsuccessful inhibitions}.
This is not a distinction without a difference for at least two reasons.
Firstly it demonstrates that inhibitory activation is invoked even when
participants had already reached the previously dubbed \textit{point of no return} and
secondly
because it demonstrates the sensitivity of response force measurements to the
underlying mechanisms and suitability of physiological measures that are
more fine-grained to investigate motor inhibition.\\

To this end, we conducted a first experiment that doesn't include any
sentential material, in order to demonstrate our ability to detect correlates
of motor inhibition in a Stop-Signal task through both behavioral and
physiological measures.
